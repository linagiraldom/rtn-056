\section{Methodology} \label{sec:methods}

The images used in this study underwent standard reduction techniques, including overscan, bias, dark, and defects reduction. Defects reduction involves creating maps that indicate regions with both bright and dark defects, such as dead pixels. These reductions serve as the starting state for the PTC study. Subsequently, other corrections, such as linearity and crosstalk, are applied to analyze their effects on the main parameters of the camera.

\vspace{3mm}
Professor Craig Lage generated supercalibrations for the entire focal plane, which are stored in his personal collection at \textit{u/cslage/calib/13144/calib.20220103}. The images shown in Figures \ref{fig:superbias} and \ref{fig:superdark} are superbias and superdarks, respectively, generated as part of the learning process for producing the calibration images using the LSST's software called \textit{DM stack}.

\vspace{3mm}
The code used to generate the calibration images via \textit{DM stack} can be found at \href{https://github.com/lsst/cp_pipe}{https://github.com/lsst/cp\_pipe}. This code generates the data for the construction of the PTC and calculates the gain for each CCD segment as its FWC. Additionally, it calculates the gain using another method, which was analyzed in Section \ref{subsec: method_gainflat} to determine the differences between it and the PTC method. The alternative method involves using two pairs of flats for gain estimation. We utilized two versions of the \textit{DM stack} code\footnote{The DM stack software is developed for the LSST and is publicly available on GitHub at \href{https://github.com/lsst/}{https://github.com/lsst/}} for the entire focal plane:

\begin{itemize}
    \item w\_2022\_27: Initial version we started working with.
    \item w\_20222\_32: Version that includes one of our main results on obtaining the gain with pairs of flats (see the \href{https://jira.lsstcorp.org/browse/DM-35790}{DM-35790} ticket in Jira).
\end{itemize}